\chapter{Introduction and Examples}
A complex manifold is a paracompact Hausdorff space which has a covering by neighbourhoods each homeomorphic to an open set in the $m$-dimensional complex number space such that where two neighborhoods overlap the local coordinates transform by a complex analytic transformation. That is, if $z^1,\cdots,z^m$ are local coordinates in one such neighborhood and if $w^1,\cdots,w^m$ are local coordinates in another neighborhood, then where they are both defined, we have 
\[w^i=w^i(z^1,\cdots,z^m),\]
where each $w^i$ is a holomorphic (or analytic) function of the $z$'s and the functional determinant 
\[\frac{\partial (w^1,\cdots,w^m)}{\partial (z^1,\cdots,z^m)}\neq 0.\]
We will give some examples of complex manifolds:

\begin{example}
The complex number space $\C^m$ whose points are the ordered $m$-tuples of complex numbers $(z^1,\cdots,z^m)$. $\C^1$ is called the Gaussian plane.
\end{example}
\begin{example}
The complex projective space $\PP^m$. To define it, take $\C^{m+1}-0$, where $0$ is the point $(0,\cdots,0)$, and identify those points $(z^0,z^1,\cdots,z^m)$ which differ from each other by a factor. The resulting quotient space is $\PP^m$. It can be covered by $m+1$ open sets $U_i$ defined respectively by 
\[\{z^i\neq 0\}, \quad 0\leq i\leq m.\] 
In $U_i$ we have the local coordinates 
\[\zeta_i^k=\frac{z^k}{z^i},\quad 0\leq k\leq m,\ k\neq i.\]
The transition of local coordinates in $U_i\cap U_j$ is given by 
\[\zeta^h_j=\frac{\zeta^h_i}{\zeta^j_i}\quad 0\leq h\leq m,\ h\neq j,\]
which are holomorphic functions.\par
In particular, $\PP^1$ is the Riemann sphere.
\end{example}